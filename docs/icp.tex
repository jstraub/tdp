\section{ICP Math}

Assume we have a model point cloud with points $p_i$ and surface
normals $n_i$ and we want to align the currently sensed point cloud
with points $q_i$ to the model by estimating the relative
transformation $(R_{mc}, t_{mc})$. The ICP point-to-plane distance
error function is given as:
\begin{align}
  \sum_i \| ( R_{mc} q_i + t_{mc} - p_i )^T n_i \|^2_2
\end{align}
Note that this already assumes a data association between model and current point cloud.
This association can be obtained efficiently in the case of RGBD frames
by projecting the current point cloud into the model camera frame and
forming the association by the pixel positions of the projected current point cloud.

\subsection{Solution using small angle assumption}
Locally, around the current relative transformation estimate $(R_{mc},
t_{mc})$ the cost function can be approximated by small perturbations $\xi$
in the tangent space $\se{3}$ around the current pose:
\begin{align}
  \sum_i \| ( R_{mc} \exp(\xi_w) q_i + t_{mc} + \xi_t - p_i )^T n_i \|^2_2
\end{align}

Slide 13 of 
\url{{http://resources.mpi-inf.mpg.de/deformableShapeMatching/EG2011_Tutorial/slides/2.1%20Rigid%20ICP.pdf}} and 
\url{https://www.comp.nus.edu.sg/~lowkl/publications/lowk_point-to-plane_icp_techrep.pdf}
show how to linearize this system assuming small rotations:
\begin{align}
  &\sum_i \| (  \exp(\xi_w) R_{mc} q_i + t_{mc} + \xi_t - p_i )^T n_i \|^2_2 \\
  =& \sum_i \| ( (I + [\xi_w]_\times) R_{mc} q_i + t_{mc} + \xi_t - p_i )^T n_i \|^2_2\\
  =& \sum_i \| (  [\xi_w]_\times R_{mc} q_i + R_{mc} q_i + t_{mc} + \xi_t - p_i )^T n_i \|^2_2\\
  =& \sum_i \| n_i^T (\xi_w \times (R_{mc} q_i)) + n_i^T R_{mc} q_i + n_i^T t_{mc} + n_i^T \xi_t -n_i^T p_i \|^2_2\\
  =& \sum_i \|  ( (R_{mc} q_i) \times n_i)^T \xi_w + n_i^T \xi_t + n_i^T ( R_{mc} q_i + t_{mc} - p_i) \|^2_2\\
  =& \|  A \xi - b \|^2_2  \,,\quad
  A = \begin{pmatrix}
    ((R_{mc} q_i) \times n_i)^T &  n_i^T \\
    \vdots & \vdots
  \end{pmatrix}\,,\;
  \xi = \begin{pmatrix} \xi_w \\ \xi_t \end{pmatrix}\,,\;
  b = \begin{pmatrix}
    n_i^T ( - R_{mc} q_i - t_{mc} + p_i) \\
    \vdots
  \end{pmatrix}
\end{align}
where we have used that the $\SO{3}$ exponential maps first order
expansion is $I + [\xi_w]_\times$ for small angular magnitude.

This can be solved in the standard approach using the pseudo inverse $A^TA$:
\begin{align}
  \xi = (A^TA)^{-1} A^Tb
\end{align}
For efficiency reasons $A$ and $b$ should never be constructed
directly. Instead we directly accumulate $A^TA$ and $A^Tb$:
\begin{align}
  A^TA = \begin{pmatrix}
    \sum_i A_{i0}^2 & \sum_i A_{i0} A_{i1} & \cdots \\
    \sum_i A_{i0} A_{i1} & \sum_i A_{i1}^2 & \cdots \\
      \vdots & \vdots & \ddots 
  \end{pmatrix}
  & 
  A^Tb = \begin{pmatrix}
    \sum_i A_{i0} b_{i} \\
    \sum_i A_{i1} b_{i} \\
      \vdots 
  \end{pmatrix}
\end{align}
Due to symmetry we only need to accumulate $21$ values ($22$ if we want
the cost function value as well).

\subsection{Alternative using gradient descent}

The optimum perturbation $\xi$ to minimize the cost function around the
current pose estimate can be found by gradient descent in the direction of derivative with
respect to $\xi$:
\begin{align}
  & \deriv{}{\xi} \sum_i \| ( R_{mc} \exp(\xi_w) q_i + t_{mc} + \xi_t - p_i )^T n_i \|^2_2\\
  =&  \sum_i 2 ( R_{mc} \exp(\xi_w) q_i + t_{mc} + \xi_t - p_i )^T n_i  \deriv{}{\xi}
  \left[ n_i^T R_{mc} \exp(\xi_w) q_i   + n_i^T \xi_t
  \right]\\
  =&  \sum_i 2 ( R_{mc} \exp(\xi_w) q_i + t_{mc} + \xi_t - p_i )^T n_i  
  \begin{pmatrix}
    - n_i^T [R_{mc} q_i]_\times  & n_i^T
  \end{pmatrix}
\end{align}
